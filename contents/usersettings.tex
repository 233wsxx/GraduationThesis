% !Mode:: "TeX:UTF-8"
% usersettings.tex
\usepackage{paralist}
\usepackage{dtklogos}
\usepackage{xcolor}
\usepackage{listings}
% Add Chapter to TOC && Remove Contents from TOC
\usepackage[chapter, nottoc]{tocbibind}
% 允许使用子图 subfloat
\usepackage{subfig}
% 允许使用图文混排
\usepackage{wrapfig}
\usepackage{physics}
% picinpar 和 picins 都无法成功实现图文混排。
% \usepackage{picinpar}
% \usepackage{picins}
% 使用 ccmap,则英文和数字都正常。如果论文重复率高的话,注释掉 ccmap ,英文和数字被映射为汉字,相当于增加干扰码,降低重复率。
\usepackage{ccmap}

\usepackage{tikz}
\usepackage{amsmath}
\usetikzlibrary{arrows.meta}
\usepackage{listofitems} % for \readlist to create arrays
\tikzstyle{mynode}=[thick,draw=black,circle,minimum size=22]
% \newtheorem{环境名}[编号延续]{显示名}[编号层次]
% 在下例中,我们定义了四个环境:定义、定理、引理和推论,它们都在一个section内编号,而引理和推论会延续定理的编号。
% \newtheorem{definition}{定义}[section]
% \newtheorem{theorem}{定理}[section]
% \newtheorem{lemma}[theorem]{引理}
% \newtheorem{corollary}[theorem]{推论}

% Define colors for code listings
\definecolor{codegreen}{rgb}{0,0.6,0}
\definecolor{codegray}{rgb}{0.5,0.5,0.5}
\definecolor{codepurple}{rgb}{0.58,0,0.82}
\definecolor{backcolour}{rgb}{0.98,0.98,0.98} % Very light gray background

% Define a simple listings style with better space utilization
\lstdefinestyle{simplestyle}{
    backgroundcolor=\color{backcolour}, % Light background
    commentstyle=\color{codegreen},     % Style for comments
    keywordstyle=\color{blue!80}\bfseries, % Style for keywords
    numberstyle=\tiny\color{codegray},  % Style for line numbers
    stringstyle=\color{codepurple},    % Style for strings
    basicstyle=\ttfamily\small,        % Basic font style
    breakatwhitespace=false,           % Break lines only at whitespace? No
    breaklines=true,                   % Enable line breaking
    captionpos=b,                      % Caption position below
    keepspaces=true,                   % Keep spaces to preserve indentation
    numbers=left,                      % Line numbers on the left
    numbersep=5pt,                     % Space between number and code
    showspaces=false,                  % Don't show spaces visually
    showstringspaces=false,            % Don't show spaces in strings visually
    showtabs=false,                    % Don't show tabs visually
    tabsize=2,                         % Tab width
    % Space saving options:
    frame=none,                        % No frame around the code
    xleftmargin=0em,                   % Left margin
    xrightmargin=0em,                % Right margin
    aboveskip=0em,                   % Space above listing
    belowskip=0em,                   % Space below listing
    columns=flexible                   % Flexible columns for better space usage
}

% To apply this style, you can either modify the \lstset command below
% to include 'style=simplestyle' or use the style explicitly, e.g.,
% \begin{lstlisting}[style=simplestyle, language=Python]
% ... code ...
% \end{lstlisting}

\lstset{language=TeX}
% \lstset{extendedchars=false}
\lstset{breaklines}	
\lstset{numbers=left,numberstyle=\tiny,%commentstyle=\color{red!50!green!50!blue!50},
frame=shadowbox,rulesepcolor=\color{red!20!green!20!blue!20},escapeinside=``,
xleftmargin=2em,xrightmargin=2em,aboveskip=1em,backgroundcolor=\color{red!3!green!3!blue!3},
basicstyle=\small\ttfamily,stringstyle=\color{purple},keywordstyle=\color{blue!80}\bfseries,
commentstyle=\color{olive}}
\numberwithin{footnote}{page}
\renewcommand{\thefootnote}{\arabic{footnote}}
\renewcommand{\CTeX}{\SHUANG{C}\TeX}
