\acknowledgment


光阴似箭,四载大学时光转瞬即逝。回首望去,这段旅程充满了探索与成长。
本论文的顺利完成,凝聚了众多师长、同学和亲友的关心与帮助,
在此谨致以最诚挚的谢意。

首先,我要特别感谢我的导师,山东大学张洋教授。
张老师不仅引领我进入了中微子探测这一引人入胜的研究领域,
为我提供了宝贵的研究机会,更在整个研究过程中给予了我悉心的指导和宝贵的支持,使我得以不断探索和进步。

同时,衷心感谢欧阳帅师兄和王菲师姐。他们在我研究起步阶段,
耐心解答了我许多基础问题,帮助我扫清了诸多障碍,
为我在SNO+实验的研究奠定了坚实基础。

在加拿大阿尔伯塔大学学习交流期间,Aksel Hallin教授、
杨绍凯老师、胡捷老师、David Auty老师以及Muhammad Mubasher同学、
Muhammad Sadegh Esmaeilian同学在学业和生活上均给予了我
无私的帮助和指导,在此一并表示感谢。

感谢SNO+合作组全体成员,在论文分析和数据准备过程中提供的支持与协助。

感谢2021级泰山学堂物理取向的全体同学。与各位并肩学习、
共同生活的点点滴滴,是我大学生涯中珍贵的记忆。我们相互扶持,
共同进步,这段同窗情谊将永远铭记于心。

最后,我要向我的父母和家人表达最深切的感谢。
你们无条件的支持、理解与关爱,是我能够专注于学业、
克服困难的最坚强后盾。

由于学识所限,文中疏漏在所难免,恳请各位老师和专家不吝赐教,批评指正。

