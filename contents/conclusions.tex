% !Mode:: "TeX:UTF-8"
\chapter{总结与展望}

\section{总结}

本论文围绕大型液体闪烁体探测器 SNO+ 实验中的事件顶点重建问题,深入研究了基于深度学习的方法,并着重探讨了其预测结果的不确定性量化 (UQ)。

首先,论文系统回顾了中微子物理学的基本知识,
包括中微子的发现、振荡现象、质量问题以及双贝塔衰变 
(特别是无中微子双贝塔衰变 $0\nu\beta\beta$) ,
为理解 SNO+ 实验的物理目标奠定了基础。接着,
介绍了深度学习的基本概念,从神经网络基础架构到
CNN、RNN,并重点阐述了在序列数据处理中表现优异的 
Transformer 模型及其核心的自注意力机制。
同时,论文详细讨论了深度学习中的不确定性量化问题,
区分了认知不确定性和偶然不确定性,
并介绍了BNN、MC Dropout和深度集成等主流 UQ 方法。

随后,论文详细介绍了 SNO+ 实验的概况,
包括其科学目标 (探测 ${}^{130}$Te 
的 $0\nu\beta\beta$) 、实验装置以及
主要的背景来源 (如 ${}^{8}$B 太阳中微子,
${}^{214}$Bi-${}^{214}$Po和
${}^{212}$Bi-${}^{212}$Po) 。
在此基础上,
论文对比了SNO+中传统的基于MLE的位置重建方法
和基于深度学习的方法。

本文的核心工作包括:设计并实现了一个基于 Transformer 架构的深度学习模型,
用于 SNO+ 实验中 ${}^{8}$B 太阳中微子事件的顶点位置重建,该模型能够利用 
PMT 的时间、位置等信息作为输入,通过自注意力机制有效捕捉被击中 PMT 
之间的复杂关联。利用包含不同探测器运行状态的SNO+模拟数据对模型进行了
训练和评估,结果表明(如图\ref{fig:position_reconstruction}所示),
该深度学习模型在位置重建性能上基本达到了传统的MLE方法。
此外,论文应用了MC Dropout对所开发的 
Transformer重建模型进行了不确定性量化,系统分析了其随能量、
径向位置等物理量的变化趋势。
% 最后,论文还探讨了粒子物理实验领域和深度学习领域中不确定性概念的异同,
% 并尝试利用深度集成方法分解认知不确定性和偶然不确定性,为理解模型预测的
% 可靠性来源提供了更深入的视角。

总之,本研究成功将 Transformer 深度学习模型应用于 SNO+ 事件重建,
并对其进行了不确定性量化,
验证了该方法在提升重建精度和提供可靠性评估方面的潜力。

\section{展望}

尽管本研究取得了一定的进展,但仍有许多地方需要进一步的探索和改进。

首先,在模型优化与扩展方面,我们可以尝试采用更好的神经网络架构,
并将 PMT 电荷信息等更多特征加入到模型输入里面来提升重建的精度,
这个方法也可以推广到能量重建、粒子鉴别等其他任务。同时,我们也要
对下一步通过捕捉被PMT直接接收的光子来进行方向重建进行探索。

在不确定性量化方面,有必要对不确定性估计进行
更严格的校准,比较更多 UQ 方法(如 BNN 的变分推断),并深入研究如何将量化
的不确定性有效应用于后续的物理分析,例如在背景抑制、信号提取或系统误差评估
等等。此外,未来还需将训练好的模型和 UQ 方法应用于 SNO+ 的真实
实验数据,检验其在实际应用中的表现,并尝试去解决模拟与真实数据之间的差异问题。
考虑到深度学习模型(尤其是 Transformer 和集成方法)的计算成本较高,
我们该需要探索其他方法以提高计算效率,便于在 SNO+ 
的数据处理流程中部署。最后,本研究开发的基于 Transformer 的重建和 UQ 
方法具有一定的通用性,未来可以探索其在其他大型粒子物理实验中的应用。

通过对上述方向的深入研究,有望进一步发挥深度学习和不确定性量化在粒子物理实验数据分析中的作用,为揭示中微子的奥秘和探索新物理提供更强大的工具。





