% !Mode:: "TeX:UTF-8"
\chapter{总结与展望}

\section{总结}

本论文围绕大型液体闪烁体探测器 SNO+ 实验中的事件顶点重建问题,深入研究了基于深度学习的方法,并着重探讨了其预测结果的不确定性量化 (UQ)。

首先,论文系统回顾了中微子物理学的基本知识,包括中微子的发现、振荡现象、质量问题以及双贝塔衰变 (特别是无中微子双贝塔衰变 $0\nu\beta\beta$) ,为理解 SNO+ 实验的物理目标奠定了基础。接着,介绍了深度学习的基本概念,从神经网络基础架构到卷积神经网络 (CNN)、循环神经网络 (RNN),并重点阐述了在序列数据处理中表现优异的 Transformer 模型及其核心的自注意力机制。同时,论文详细讨论了深度学习中的不确定性量化问题,区分了认知不确定性和偶然不确定性,并介绍了贝叶斯神经网络 (BNN)、蒙特卡洛 Dropout (MC Dropout) 和深度集成等主流 UQ 方法。

随后,论文详细介绍了 SNO+ 实验的概况,包括其科学目标 (探测 ${}^{130}$Te 的 $0\nu\beta\beta$) 、实验装置以及主要的背景来源 (如 ${}^{8}$B 太阳中微子,${}^{214}$Bi-${}^{214}$Po和${}^{212}$Bi-${}^{212}$Po) 。在此基础上,论文对比了 SNO+ 中传统的基于最大似然估计 (MLE) 的位置重建方法和基于深度学习的方法。

本文的核心工作在于:
\begin{itemize}
    \item 设计并实现了一个基于 Transformer 架构的深度学习模型,用于 SNO+ 实验中 ${}^{8}$B 太阳中微子事件的顶点位置重建。该模型利用 PMT 的时间、位置等信息作为输入,通过自注意力机制捕捉被击中的 PMT 之间的复杂关联。
    \item 使用包含不同探测器运行状态的 SNO+ 模拟数据对模型进行了训练和评估。结果表明 (如图\ref{fig:position_reconstruction}所示) ,该深度学习模型在位置重建精度上优于传统的 MLE 方法,尤其在低能量区域表现更稳定。
    \item 应用了 MC Dropout 和深度集成两种方法对所开发的 Transformer 重建模型进行了不确定性量化。分析了预测不确定性与实际重建误差的关系、不确定性的校准情况以及其随能量、径向位置等物理量的变化趋势。
    \item 探讨了粒子物理实验领域和深度学习领域中不确定性概念的异同,并尝试利用深度集成方法分解认知不确定性和偶然不确定性,为理解模型预测的可靠性来源提供了更深入的视角。
\end{itemize}

总之,本研究成功将 Transformer 深度学习模型应用于 SNO+ 事件重建,并对其进行了不确定性量化,验证了该方法在提升重建精度和提供可靠性评估方面的潜力。

\section{展望}

尽管本研究取得了一定的进展,但仍有许多值得进一步探索的方向:
\begin{itemize}
    \item \textbf{模型优化与扩展:} 可以尝试更先进的神经网络架构 (如图神经网络) ,或将 PMT 电荷信息等更多特征融入模型输入,以期进一步提升重建精度。同时,可以将该方法扩展到能量重建、粒子鉴别等其他重建任务。
    \item \textbf{不确定性量化深化:} 需要对不确定性估计进行更严格的校准研究。可以比较更多 UQ 方法 (如 BNN 的变分推断) 的效果。更重要的是,需要研究如何将量化的不确定性有效地应用于后续的物理分析中,例如在背景抑制、信号提取或系统误差评估中利用 UQ 信息。
    \item \textbf{真实数据验证:} 目前的研究基于模拟数据,未来需要将训练好的模型和 UQ 方法应用于 SNO+ 的真实实验数据,检验其在实际应用中的表现,并解决可能存在的模拟与真实数据之间的差异问题。
    \item \textbf{计算效率与部署:} 深度学习模型 (尤其是 Transformer 和集成方法) 的训练和推理计算成本较高,需要研究模型压缩、知识蒸馏或硬件加速等技术,以提高计算效率,便于在 SNO+ 的数据处理流程中进行部署。
    \item \textbf{推广应用:} 本研究中开发的基于 Transformer 的重建和 UQ 方法具有一定的通用性,未来可以探索将其应用于其他类似的大型粒子物理实验中。
\end{itemize}

通过对上述方向的深入研究,有望进一步发挥深度学习和不确定性量化在粒子物理实验数据分析中的作用,为揭示中微子的奥秘和探索新物理提供更强大的工具。





