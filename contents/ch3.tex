\chapter{深度学习以及其不确定性量化简介}

深度学习通过构建多层非线性变换网络实现对数据特征的层次化抽象,
已成为现代人工智能的核心范式。本章在\ref{intro-NN}节中介绍了神经网络的基础架构,
阐述深度学习的基础理论框架,重点解析神经网络的结构特性与训练机制,
在\ref{transformer}节深入探讨了Transformer模型的自注意力机制,
并在\ref{uncertainties}节全面论述了不确定性量化的前沿方法与技术挑战。

\section{神经网络基础架构简介}\label{intro-NN}

神经网络是一种模拟人脑神经元结构和功能的计算模型。它由大量的神经元通过连接权重相互连接而成。
神经网络的基本单元是神经元,每个神经元接收来自其他神经元的输入信号,并通过激活函数将其转换为输出信号。
神经网络的学习过程就是通过调整连接权重来最小化预测值与真实值之间的误差。
神经网络的结构通常分为输入层、隐藏层和输出层。输入层接收输入数据,隐藏层通过激活函数对输入数据进行非线性变换,
输出层将隐藏层的输出转换为最终的预测结果。
神经元的输出信号可以表示为:
\begin{equation}
  a_{i,j} = f\left(\sum_{k=1}^{n} w_{i,j,k} x_k + b_{i,j}\right)
\end{equation}
其中,$a_{i,j}$为第$i$个神经元在第$j$层的输出信号,
$w_{i,j,k}$为第$i$个神经元在第$j$层与第$k$个神经元在第$j+1$层的连接权重,
$b_{i,j}$为第$i$个神经元在第$j$层的偏置项,$f(\cdot)$为激活函数,用于引入非线性变换。
常用的激活函数有sigmoid函数、反正切(tanh)函数和ReLU(rectified linear unit,整流线型单元)函数等。
其中sigmoid和tanh函数呈S型曲线,特别适合二分类问题;而ReLU函数则是一种分段线性函数,在多分类问题中表现优异。
图\ref{NNstructure}展示了一个简单的神经网络结构,其中包含一个输入层、两个隐藏层和一个输出层。

\begin{figure}[htbp]
  \centering
  \begin{tikzpicture}[x=2cm,y=2cm] % 调整了y的值以增加垂直间距
    \usetikzlibrary{fit}
    \readlist\Nnod{2,3,3,2} % 输入层2个node,两个隐藏层每层3个node,输出层2个node
    \foreachitem \N \in \Nnod { % loop over layers
      \foreach \i [evaluate={\x=\Ncnt; \y=\N/2-\i+0.5; \prev=int(\Ncnt-1);}] in {1,...,\N} { % loop over nodes
      % 根据不同层添加不同的标签
      \ifnum\Ncnt=1
        \node[mynode, fill=blue!40](N\Ncnt-\i)at(\x,\y){$x_{\i}$};
      \else
        \ifnum\Ncnt=\Nnodlen
        \node[mynode, fill=red!40](N\Ncnt-\i)at(\x,\y){$y_{\i}$};
        \else
        \node[mynode, fill=green!40](N\Ncnt-\i)at(\x,\y){$a_{\the\numexpr\Ncnt-1\relax,\i}$};
        \fi
      \fi
      % 为连接的线加上标签,只在输入层到第一个隐藏层中,当连接来自x1时添加对应数字的权重标签
      \ifnum\Ncnt>1
        \pgfmathsetmacro{\prevCount}{\Nnod[\prev]}
        \foreach \j in {1,...,\prevCount} {
        \ifnum\Ncnt=2
          \ifnum\i=1
          \ifnum\j=1
            \draw[thick](N\prev-\j)-- node[midway, above] {$\omega_{1,1,1}$}(N\Ncnt-\i);
          \else
            \draw[thick](N\prev-\j)--(N\Ncnt-\i);
          \fi
          \else
          \draw[thick](N\prev-\j)--(N\Ncnt-\i);
          \fi
        \else
          \draw[thick](N\prev-\j)--(N\Ncnt-\i);
        \fi
        }
      \fi
      }
      % 绘制层的虚线边框
      \ifnum\Ncnt=1
        \def\LayerColor{black}
      \else
        \ifnum\Ncnt=\Nnodlen
          \def\LayerColor{black}
        \else
          \def\LayerColor{black}
        \fi
      \fi
      \node[draw=\LayerColor, dashed, rounded corners, inner sep=5pt, fit=(N\Ncnt-1)(N\Ncnt-\N)] {};
      % 添加层的标签
      \ifnum\Ncnt=1
      \node[below] at(\Ncnt,-1.5){输入层};
      \else
      \ifnum\Ncnt=\Nnodlen
        \node[below] at(\Ncnt,-1.5){输出层};
      \else
        \node[below] at(\Ncnt,-1.5){隐藏层};
      \fi
      \fi
    }
    \end{tikzpicture}
  \caption{神经网络基本结构的示意图,这里包含输入层、两个隐藏层和输出层,每个隐藏层的神经元都被连接到前一层的所有神经元。
  最后的输出层将隐藏层的输出转换为最终的预测结果,在训练过程中,最终的输出结果会与真实值进行比较,从而计算损失函数。}
  \label{NNstructure}
\end{figure}

神经网络的训练过程通常采用反向传播算法,该算法通过计算损失函数的梯度来更新连接权重。损失函数用于量化预测值与真实值之间的差距,
常见的损失函数包括均方误差和交叉熵等。训练过程中,通常使用梯度下降算法不断迭代优化这些权重,以最小化损失函数并找到最优参数,
其数学本质为参数空间的最速下降过程:

\begin{equation}
  \theta = \arg\min_{\theta} \mathcal{L}(f_\theta(x), y)
\end{equation}
其中,$\theta$为神经网络的参数,$\mathcal{L}$为损失函数,$f_\theta(x)$为神经网络的预测函数,$y$为真实值。


上文介绍的神经网络属于前馈神经网络,
基于前馈神经网络的深度学习模型有很多种,最常见的有卷积神经网络(CNN,convolutional neural network)和循环神经网络(RNN,recurrent neural network)。
CNN是一种特殊的前馈神经网络,主要用于处理图像数据。它通过卷积层和池化层提取图像特征,并通过全连接层进行分类。
卷积层通过卷积操作提取局部特征,池化层通过下采样操作减少特征图的尺寸,从而降低计算复杂度和防止过拟合。
RNN是一种特殊的前馈神经网络,主要用于处理序列数据。它通过循环结构捕捉序列数据中的时序关系,
并通过LSTM(long short-term memory,长短时记忆)\cite{hochreiter1997long}
或GRU(gated recurrrent unit,门控循环单元)\cite{DBLP:journals/corr/ChoMGBSB14}等结构解决长依赖问题。
卷积神经网络和循环神经网络都是深度学习的重要组成部分,不过对于传统的RNN,
由于其在长序列数据上的训练效率较低,且容易出现梯度消失或梯度爆炸等问题,因此在处理长序列数据时,除了
LSTM和GRU等变种RNN外,接下来要介绍的Transformer模型也被广泛应用。

\section{Transformer}\label{transformer}

Transformer是一种基于自注意力机制的神经网络模型,最早由Google在2017年提出\cite{vaswani2017attention}。Transformer模型的主要特点是使用自注意力机制来捕捉输入序列中不同位置之间的依赖关系,从而实现对序列数据的建模。
Transformer模型的基本结构包括编码器和解码器两个部分。编码器将输入序列转换为一个上下文向量,解码器根据上下文向量生成输出序列。编码器和解码器都由多个相同的层堆叠而成,每一层都包含多头注意力机制和前馈神经网络。
编码器和解码器之间通过一个交叉注意力机制进行连接,从而实现对输入序列和输出序列之间的依赖关系建模。
编码器的输入是一个序列,输出是一个上下文向量。解码器的输入是上下文向量和前一个时刻的输出,输出是当前时刻的预测结果。编码器和解码器之间通过一个交叉注意力机制进行连接,从而实现对输入序列和输出序列之间的依赖关系建模。
编码器的输入序列经过多个多头注意力机制和前馈神经网络的处理后,输出一个上下文向量。解码器的输入是上下文向量和前一个时刻的输出,经过多个自注意力机制和前馈神经网络的处理后,输出当前时刻的预测结果。
解码器的输出序列经过一个线性变换和softmax函数的处理后,得到每个位置的预测概率分布。
解码器的输出序列可以通过采样或贪心搜索等方法生成最终的输出序列。
图\ref{transformer}是Transformer结构的示意图。

\begin{figure}[htbp]
  \centering
  \includegraphics[width=0.45\textwidth]{figures/transformer.png}
  \caption{Transformer模型的基本结构示意图\cite{vaswani2017attention}}
  \label{transformer}
\end{figure}

\subsection{注意力机制}
注意力机制是一种用于捕捉输入序列中不同位置之间的依赖关系的机制。
注意力机制的基本思想是通过计算输入序列中每个位置与其他位置之间的相似度来生成一个上下文向量,从而实现对序列数据的建模。
注意力机制的计算过程包括三个步骤: 计算注意力权重、计算上下文向量和计算输出。


计算注意力权重是通过计算输入序列中每个位置与其他位置之间的相似度来生成一个注意力权重矩阵。
注意力权重矩阵的每一行表示当前时刻的输入位置与其他位置之间的相似度,每一列表示其他位置对当前时刻的输入位置的影响程度。
注意力权重矩阵的计算公式为\cite{vaswani2017attention}:
\begin{equation}
  \label{attention}
  \text{Attention} = \text{softmax}\left(\frac{QK^T}{\sqrt{d_k}}\right)
\end{equation}
其中,$Q$为查询(Query)矩阵,$K$为键(Key)矩阵,$d_k$为键矩阵的维度。
注意力权重矩阵的每一行表示当前时刻的输入位置与其他位置之间的相似度,每一列表示其他位置对当前时刻的输入位置的影响程度。
上面分析的注意力机制被称为“缩放点积注意力(Scaled dot-product attention)”,不过我们从图\ref{transformer}可以看出,
进行应用的是“多头注意力(Multi-head attention)”。这是由多个缩放点积注意力经过投影$h$次然后经过
$d_k$,$d_k$和$d_v$并行运算最后叠加而成的。这一机制使得模型可以结合多个不同投影子空间的表现,从而获得更高精确度的预测。
为这两种注意力机制的示意图由图\ref{fig:multi-head-att}给出,其中缩放点积注意力由公式\ref{attention},而多头注意力用公式表征如下\cite{vaswani2017attention}:

\begin{align}
  \mathrm{MultiHead}(Q, K, V)&= \mathrm{Concat}(\mathrm{head_1}, ..., \mathrm{head_h})W^O\\
  \text{其中}~\mathrm{head_i} &= \mathrm{Attention}(QW^Q_i, KW^K_i, VW^V_i)\\
\end{align}

其中这些投影为参数矩阵$W^Q_i \in \mathbb{R}^{d_{model} \times d_k}$, $W^K_i \in \mathbb{R}^{d_{model} \times d_k}$, $W^V_i \in \mathbb{R}^{d_{model} \times d_v}$ and $W^O \in \mathbb{R}^{hd_v \times d_{model}}$.
\begin{figure}[htbp]
  \centering
  \begin{minipage}[t]{0.4\textwidth}
    \centering
    \vspace{0.5cm}
    \includegraphics[scale=0.6]{figures/dotproductattention.png}
  \end{minipage}
  \begin{minipage}[t]{0.4\textwidth}
    \centering
    \vspace{0.1cm}
    \includegraphics[scale=0.6]{figures/multihead.png}
  \end{minipage}
  \caption{左图为缩放点积注意力,右图为多头注意力由多个并行运行的缩放点积注意力层组成。\cite{vaswani2017attention}}
    \label{fig:multi-head-att}
\end{figure}

总而言之,Transformer模型通过其核心的自注意力机制,特别是多头注意力变体,
有效地捕捉了序列数据中的长距离依赖关系。其编码器-解码器架构,
结合位置编码和前馈网络,使其在序列到序列任务中取得了显著成功,
并成为现代深度学习模型的重要基石。


\section{深度学习中的不确定性分析}\label{uncertainties}
深度学习模型在各种领域展现出了强大的预测能力,
然而对这些模型预测结果的不确定性进行量化是一个重要且具有挑战性的问题。
不确定性量化(Uncertainty Quantification,UQ)旨在评估和表征
深度学习模型预测的可靠性和置信度。在实际应用中,
了解模型预测的不确定性对于决策制定、风险评估和模型解释至关重要。

在深度学习领域,不确定性可分为认知不确定性(模型不确定性)和随机不确定性(数据不确定性)。\cite{ABDAR2021243}
前者源于模型参数的不确定性,可通过贝叶斯方法量化;后者源于数据本身的噪声或随机性,
通常通过概率分布建模。深度学习中的不确定性量化方法主要包括BNN
(Bayesian neural network,贝叶斯神经网络)、
深度集成方法\cite{NIPS2017_9ef2ed4b}和MC(Monte Carlo,蒙特卡洛) dropout等技术。
这些方法使我们能够对模型预测结果提供可靠的不确定性估计,
从而增强模型在实际应用中的可靠性和可解释性。

\subsection{BNN}
BNN 的核心思想是将神经网络的权重($\omega$)和偏置视为概率分布,
而不是固定的点估计值,如图\ref{bnn}所示。其目标是根据观测数据$D$推断权重的后验分布 $p(w|D)$ 。
对于新的输入 $x^*$,
BNN 的预测是通过对所有可能的权重配置进行边缘化(或积分)得到的,
即进行贝叶斯模型平均(Bayesian model averaging):\cite{ABDAR2021243}

\begin{equation}
    p(y^*|x^*,D) = \int p(y^*|x^*,w)p(w|D)dw
\end{equation}

\begin{figure}[htbp]
    \centering
    \includegraphics[width=0.45\textwidth]{figures/Bayesianneuralnetwork.pdf}
    \caption{贝叶斯神经网络的基本结构示意图\cite{DBLP:journals/corr/abs-2011-06225}}
    \label{bnn}
\end{figure}
由于精确计算后验分布通常是不可行的,我们一般
需要采用一些近似推断方法,
例如VI(Variational Inference,变分推断),
MCMC(Markov Chain Monte Carlo,马尔可夫链蒙特卡洛),拉普拉斯近似
等。

对于VI,我们使用一个简单的、可处理的概率分布
$q(\theta)$(例如,均值场高斯分布)来近似真实的后验分布
$p(w|D)$,通过最小化它们之间的Kullback-Leibler散度\cite{1320776d-9e76-337e-a755-73010b6e4b64}来优化近似分布的参数
。VI在计算上通常比MCMC更高效,但其结果依赖于近似分布的选择。

对于MCMC,我们通过构建马尔可夫链来从后验分布中采样。
MCMC一般被认为能够提供更加精确的估计,
但计算成本非常高,尤其对于参数量巨大的深度神经网络。

对于拉普拉斯近似,我们将后验分布近似为以最大后验概率
(MAP)估计为中心的高斯分布,其协方差矩阵由损失函数的
Hessian矩阵的逆给出。

BNN能够自然地捕捉认知不确定性(通过权重分布的方差体现)。
如果模型被设计为预测输出的分布参数
(例如,预测高斯分布的均值和方差),则也可以捕捉偶然不确定性。
一些研究工作尝试利用 BNN 来分离统计不确定性和系统不确定性。
一方面,BNN提供了严谨的贝叶斯框架;
有潜力产生良好校准的不确定性估计;
能够捕捉模型参数之间的复杂相关性。
不过另一方面,BNN计算成本高昂(尤其是MCMC);
结果可能依赖于先验分布的选择;而
VI的近似质量可能受限;训练过程可能不稳定。

\subsection{MC Dropout}

Dropout是一种常用的正则化技术,在训练过程中以一定概率随机“丢弃”(即设置为零)
神经元的输出。MC dropout将这一过程扩展到测试(推断)阶段: 对同一个输入样本,
通过多次随机前向传播,每次使用不同的随机Dropout掩码(mask),
从而得到一组不同的预测结果,如图\ref{mc_dropout}所示。


\begin{figure}[htbp]
    \centering
    \includegraphics[width=0.45\textwidth]{figures/MonteCarloDropout.pdf}
    \caption{MC dropout的基本结构示意图\cite{DBLP:journals/corr/abs-2011-06225}}
    \label{mc_dropout}
\end{figure}

MC dropout主要捕捉认知不确定性(模型不确定性),这种不确定性体现在
不同dropout掩码下预测结果的变化性上。它可以与预测偶然不确定性的模型结合使用。
不过相比BNN,MC dropout实现更简单,可以方便地应用于已经使用
Dropout进行正则化的现有网络架构;计算成本相比BNN更低。
然而,其贝叶斯近似的质量可能有限;对于大型复杂模型,其不确定性估计可能校准不佳
或不足;而且其性能(包括UQ质量)依赖于dropout率的选择和网络结构;
不确定性的解释可能不够直接。

\subsection{深度集成}

深度集成是训练多个(N个)结构相同(或相似)但参数独立的神经网络模型。\cite{NIPS2017_9ef2ed4b}
这些模型通常使用不同的随机权重初始化,有时也使用不同的训练数据子集
(例如通过bootstrapping,如图\ref{boosttrap}所示)。进行训练。最终预测结果通过组合N个模型的输出得到
(例如,对回归任务取均值,对分类任务取平均概率),而不确定性则由模型预测结果之间的差异
(例如方差)来估计。

\begin{figure}
    \centering
    \includegraphics[width=0.45\textwidth]{figures/BootstrapModel.pdf}
    \caption{深度集成训练中Bootstrapping的基本结构示意图\cite{DBLP:journals/corr/abs-2011-06225}}
    \label{boosttrap}
\end{figure}

该方法主要通过模型多样性来捕捉认知不确定性。
可以与能够预测偶然不确定性的模型(例如,预测输出分布的均值和方差)结合,
使用集成预测的方差来估计认知不确定性部分。

深度集成概念简单直观;在许多基准测试和实际应用中,常常能达到顶尖的预测精度和良好的校准性能;
模型训练过程可以并行化;但是训练和推断的计算成本很高(需要训练和运行N个模型),需要大量的存储空间;
同时我们还需要确保集成成员具有足够的多样性;仅仅通过集成预测的差异来估计不确定性,
缺乏严格的数学基础来保证其能可靠地同时涵盖偶然和认知不确定性。

总而言之,不确定性量化在深度学习模型的应用中具有重要意义,
有助于提升模型的可靠性和决策质量。
选择合适的不确定性量化方法需要在计算成本和精度之间进行权衡。

\section{本章小结}

本章首先介绍了深度学习的基础,从神经网络的基本结构、神经元、激活函数和训练过程入手,
阐述了前馈神经网络的基本原理,并简要提及了CNN和RNN等常见变种。
随后,重点介绍了在序列数据处理中表现优异的Transformer模型,
详细解释了其核心的自注意力机制,特别是缩放点积注意力和多头注意力。
最后,本章深入探讨了深度学习中的不确定性量化问题,
区分了认知不确定性和随机不确定性,并详细介绍了三种主流的不确定性量化方法:
贝叶斯神经网络(BNN)、蒙特卡洛Dropout(MC Dropout)和深度集成。
对每种方法的原理、优缺点和适用场景进行了分析,
强调了不确定性量化对于提升模型可靠性和可解释性的重要性。

