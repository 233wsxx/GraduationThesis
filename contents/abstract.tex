% !Mode:: "TeX:UTF-8"
\begin{cnabstract}
萨德伯里中微子观测站+是开创先河的萨德伯里中微子观测站的后续项目。
它位于加拿大萨德伯里地下约2千米处。它的主要物理目标是探测一个
极其罕见的事件——无中微子双贝塔衰变。
如果这一现象被观测到,它将为中微子物理和基本粒子理论
提供深远的启示。为了实现这一目标,该实验使用
一个装有780吨液体闪烁体的直径约12米的丙烯酸容器
探测器,该探测器随后将被装载1.3吨的同位素碲-130。
精确的包括事件位置,方向,时间,能量的事件重建对于区分
无中微子双贝塔衰变和背景事件的信号至关重要。
中微子实验中的传统事件重建通常使用数值方法,并且基于
最大似然估计。尽管这些方法被广泛地使用,但是当我们处理
复杂的事件,大量的噪声或者大量的数据时,这些方法
将会面临很大的局限,导致重建精度受限和计算效率低下。
不过,深度学习的方法凭借其强大的捕捉复杂,非线性
关系的能力在粒子物理领域尤其是事件重建中
显现出了其对事件重建性能提升的潜力。

尽管深度学习方法取得了令人期待的成就,但是
评估模型的预测的可靠性和鲁棒性仍然是一个重要的
挑战。特别是需要全面的不确定性量化来确保基于深度学习
的方法的重建的可信度。对于不确定性的量化,包括数据驱动的
和模型相关的不确定性,不仅对于控制预测的置信度以防止过拟合
和欠拟合很有必要,而且对于后续的物理分析也很重要。

在本论文中,我们提出了一种专为SNO+实验设计全新的深度学习驱动的事件重建方法。
这一模型采用基于生成预训练Transformer结构的神经网络架构。
为了训练和验证我们的模型,我们使用了依托实验反应堆分析工具的
蒙特卡洛模拟数据。此外,我们还引入了一种利用液体闪烁体中自然放射性同位素
铋-214和钋-214的事件验证新方法。这种方法不仅稳定可靠,还能实现基于
数据的高精度重建验证。

经过系统评估,我们的结果显示,即使训练时间较短
,所开发的深度学习模型也能在重建精度、计算效率,以及对探测器噪声和
不确定性表现出令人满意的鲁棒性。更值得注意的是,通过
蒙特卡洛Dropout技术,我们能够同时对数据驱动与模型驱动的不确定性
进行估计,从而深入理解事件重建的模型置信度。

总体而言,本研究提出并验证了一种融合不确定性估算的新型深度学习重建方法,
同时为中微子物理实验中的数据分析提供了新的思路。
这一方法不仅适用于SNO+等实验,
还可以根据需要,应用于其他中微子或稀有事件探测实验。



\cnkeywords{双贝塔衰变;中微子物理;重建;深度学习;不确定性估计;液体闪烁体探测器}

\end{cnabstract}

\begin{enabstract}

The Sudbury Neutrino Observatory+ experiment is the successor 
to the pioneering Sudbury 
Neutrino Observatory. It is located approximately 2km 
underground in Sudbury, Canada. Its primary physics goal is 
to detect a very rare event, neutrinoless double-beta 
decay. This phenomenon would provide profound 
implications for neutrino physics and fundamental particle theory 
if observed. To achieve this goal, the experiment employs a 
detector filled with 780 tonnes of liquid scintillator, which 
will subsequently be loaded with approximately 1.3 tonnes of 
the isotope $^{130}\mathrm{Te}$ in its 12-meter diameter acrylic 
vessel. Precise event reconstruction, including accurate 
determination of event location, direction, timing, and energy, 
is crucial for distinguishing the subtle signals of neutrinoless double-beta 
decay from background events.

Traditionally, event reconstruction in neutrino experiments 
usually implement numerical methods, and often based on 
maximum likelihood estimation. 
Although it's widely employed, these methods has a high probability 
to face limitations when dealing with complex events, substantial 
noise, or large datasets, leading to constrained reconstruction 
accuracy and computational inefficiencies. However, 
deep learning techniques have emerged prominently especially for
the event reconstruction task
in the field of particle physics, which has demonstrated significant 
potential for enhancing reconstruction performance due to 
their exceptional capability of capturing intricate, nonlinear 
relationships from large-scale datasets.

Despite the promising achievements brought by deep learning approaches, 
assessing the reliability and robustness of model predictions 
remains a critical challenge. Specifically, comprehensive 
uncertainty quantification is required to ensure the scientific 
credibility of deep learning–based position reconstruction algorithms. 
Quantifying uncertainties, including aleatoric (data-driven) and epistemic (model-related) uncertainties, 
is essential not only for controling the confidence of predictions to avoid overfitting or underfitting, 
but also for the subsequent physical analysis.

In this thesis, we present a novel deep learning–based event 
reconstruction algorithm which is speciallized 
to fit the unique 
characteristics of the experiment. The proposed model 
implements Generative Pre-trained Transformer-based architectures. 
To train and validate our deep learning model, we have used Monte Carlo 
simulation data based on the reactor analysis tool in the experiment.
Moreover, we introduced an innovative validation method based on 
naturally occurring radioactive 
$^{214}\mathrm{Bi}$-$^{214}\mathrm{Po}$ coincidence events within 
the scintillator. This method offers a robust, data-driven 
validation approach to evaluate the reconstruction accuracy of 
our model.

Through systematic assessments, our results 
demonstrate that the developed deep learning model 
with little training ofa relatively small dataset 
can reach almost the same performance traditional maximum likelihood 
estimation-based methods in 
terms of reconstruction precision, computational efficiency, 
and robustness to detector noise and uncertainties. Importantly, 
the implementation of Monte Carlo Dropout techniques has 
shown the potential to have  
simultaneous quantification of both aleatoric and epistemic 
uncertainties, providing comprehensive insight into the 
confidence levels of reconstruction results.

Overall, this research proposes and validates an 
innovative deep learning-based reconstruction framework 
with integrated uncertainty estimation, and contributes to 
advancing data analysis methodologies in neutrino physics experiments. 
The methodologies developed here have broader implications and can 
be adapted to other neutrino or rare-event detection experiments 
later after modification.

\enkeywords{Double-beta decay,Neutrino physics,Reconstruction,Deep Learning,Uncertainty estimation,Liquid scintillator detector}

\end{enabstract}

