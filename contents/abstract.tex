% !Mode:: "TeX:UTF-8"
\begin{cnabstract}

萨德伯里中微子观测站+实验是开创性
的萨德伯里中微子观测站的后继实验,坐落于加拿大萨德伯里地下约2公里处。
它的主要科学目标是探测极为罕见的无中微子双贝塔衰变。
如果观测到这一现象,将对中微子物理和基本粒子理论产生深远的影响。
为实现这一雄心勃勃的目标,SNO+在一个直径12米的丙烯酸容器中装载了
780吨液体闪烁体,并计划向其中加入约1.3吨的同位素
碲-130。要从背景事例中分辨出无中微子双贝塔衰变
的微弱信号,准确的事例重建(包括对事例位置、方向、时间以及能量的精确测定)
至关重要。

在传统的中微子实验中,事例重建通常依赖基于最大似然估计的数值计算方法。
虽然这些方法应用广泛,但在面对复杂的事例、大量噪声或庞大数据时,
它们往往受到内在限制,表现出重建精度不足及计算效率较低等问题。近年来,
深度学习技术在粒子物理研究中崭露头角,凭借其从大规模数据中捕捉
复杂非线性关系的卓越能力,为提升重建性能提供了重要契机。

然而,尽管深度学习方法取得了显著成就,评估模型预测的可靠性和鲁棒性
仍是一个关键挑战。尤其需要关注全面的不确定性量化,以确保基于深度学习
的重建算法在科学上具有可信度。对不确定性的定量分析(包括源于数据的随机
不确定性和源于模型的认知不确定性)不仅能够
避免预测过度自信或信心不足的问题,也对后续的物理分析至关重要。

在本论文中,我们提出了一种专门针对SNO+实验特点而设计的全新深度学习事例
位置重建算法。该模型采用了基于生成式预训练Transformer的神经网络架构,因其在处理序列
数据和注意力机制方面具有强大的能力而备受关注。为训练和验证该深度学习
模型,我们使用了基于SNO+实验中反应堆分析工具
软件的蒙特卡洛模拟数据,该数据能较好地反映真实探测器环境和事例分布。
此外,我们还提出了一种新的基于闪烁体中自然放射性的铋-214-钋-214事例符合信号
的验证方法。该方法提供了一个由数据驱动的有力验证手段,
用于直观地评估模型在空间和时间上的重建准确度,从而与传统的基于模拟的
评估形成互补。

通过系统的评价,我们的结果显示,这种基于生成式预训练Transformer的深度学习模型
在位置重建精度、计算效率以及对探测器噪声与不确定性的鲁棒性方面,均基本达到了
传统的MLE方法。更为重要的是,通过引入蒙特卡洛Dropout技术,能够同时
量化随机和认知不确定性,从而全面表征重建结果的可信度。

总体而言,本研究不仅提出并验证了一种结合不确定性估计的新型深度学习位置
重建框架,而且对中微子物理实验的数据分析方法做出了一定贡献。
所提出的方法具有广泛适用性,后续可以通过修改迁移到其他中微子或稀有事例
探测实验中。

\cnkeywords{双贝塔衰变;中微子物理;重建;深度学习;不确定性估计;液体闪烁体探测器}

\end{cnabstract}

\begin{enabstract}

The SNO+ experiment is the successor to the pioneering Sudbury 
Neutrino Observatory. It is located approximately 2km 
underground in Sudbury, Canada. Its primary scientific goal is 
to detect the exceptionally rare event, neutrinoless double-beta 
decay. This phenomenon would provide profound 
implications for neutrino physics and fundamental particle theory 
if observed. To achieve this ambitious goal, SNO+ employs a 
detector filled with 780 tonnes of liquid scintillator, which 
will subsequently be loaded with approximately 1.3 tonnes of 
the isotope $^{130}\mathrm{Te}$ in its 12-meter diameter acrylic 
vessel. Precise event reconstruction, including accurate 
determination of event location, direction, timing, and energy, 
is crucial for distinguishing the subtle signals of neutrinoless double-beta 
decay from background events.

Traditionally, event reconstruction in neutrino experiments 
relies on analytical methods, often based on maximum likelihood 
estimation. Although widely employed, these methods face 
inherent limitations when dealing with complex events, substantial 
noise, or large datasets, leading to constrained reconstruction 
accuracy and computational inefficiencies. Recently, 
deep learning techniques have emerged prominently 
within particle physics research, demonstrating significant 
potential for enhancing reconstruction performance due to 
their exceptional capability of capturing intricate, nonlinear 
relationships from large-scale datasets.

Despite the promising achievements brought by deep learning approaches, 
assessing the reliability and robustness of model predictions 
remains a critical challenge. Specifically, comprehensive 
uncertainty quantification is required to ensure the scientific 
credibility of deep learning–based position reconstruction algorithms. 
Quantifying uncertainties, including aleatoric (data-driven) and epistemic (model-related) uncertainties, 
is essential not only for controling the confidence of predictions to avoid overconfident or underconfident, 
but also for the subsequent physical analysis.

In this thesis, we present a novel deep learning–based event 
reconstruction algorithm tailored explicitly to the unique 
characteristics of the SNO+ experiment. The proposed model 
leverages GPT (Generative Pre-trained Transformer) -based neural network architectures, 
which is well-known for their strong capability in handling 
sequential and attention-based data processing tasks. To 
train and validate our deep learning model, we utilized Monte Carlo 
simulation data based on the reactor analysis tool in SNO+, 
reflecting realistic detector conditions and event distributions. 
Moreover, we introduced an innovative validation method based on 
naturally occurring radioactive 
$^{214}\mathrm{Bi}$-$^{214}\mathrm{Po}$ coincidence events within 
the scintillator. This method offers a robust, data-driven 
validation approach to intuitively evaluate the spatial and 
temporal reconstruction accuracy of our model, complementing 
traditional simulation-based evaluations.

Through systematic assessments, our results 
demonstrate that the developed Transformer-based deep learning model 
significantly outperforms traditional MLE-based methods in 
terms of reconstruction precision, computational efficiency, 
and robustness to detector noise and uncertainties. Importantly, 
the implementation of Monte Carlo Dropout techniques enabled the 
simultaneous quantification of both aleatoric and epistemic 
uncertainties, providing comprehensive insight into the 
confidence levels of reconstruction results.

Overall, this research not only proposes and validates an 
innovative deep learning-based reconstruction framework with integrated 
uncertainty estimation but also contributes to 
advancing data analysis methodologies in neutrino physics experiments. 
The methodologies developed here have broader implications and can 
be adapted to other neutrino or rare-event detection experiments 
later after modification, marking an essential step toward 
leveraging artificial intelligence technologies to advance 
frontier research in particle physics.

\enkeywords{Double-beta decay,Neutrino physics,Reconstruction,Deep Learning,Uncertainty estimation,Liquid scintillator detector}

\end{enabstract}

