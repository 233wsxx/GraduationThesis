% !Mode:: "TeX:UTF-8"
\chapter{前言}

\section{研究背景}

中微子是粒子物理学研究的热点之一。它也是标准模型中最轻的基本粒子之一,
具有极小的质量,且与物质的相互作用非常微弱,因此它们在宇宙中的传播几乎不受阻碍。
中微子振荡现象的发现,表明了中微子具有非零质量,并且不同味道的中微子之间可以相互转化,
这一发现对粒子物理学和宇宙学都产生了深远的影响,也是目前确凿的超出标准模型的物理现象。
日本的超级神冈(Super-Kamiokande)和加拿大SNO (Sudbury Neutrino Observertory,萨德伯里中微子观测站) 
实验等一系列实验,验证了中微子振荡现象\cite{PhysRevLett.87.071301}\cite{PhysRevLett.81.1562},
并测量了中微子的质量差和混合角。这些实验结果为我们理解中微子的性质提供了重要的实验依据。
而中微子仍旧是存在未解之谜,许多问题仍然悬而未决。例如,中微子的质量是多少?
中微子是否是其自身的反粒子,即中微子是否是马约拉纳粒子?这些问题的解决将有助于我们更
深入地理解基本粒子的性质和宇宙的演化。

SNO实验在完成了太阳中微子振荡的测量后,
为了以提高探测器的灵敏度和分辨率进行了一系列升级改进。
SNO+实验\cite{andringa2016current}\cite{albanese2021sno+}
是SNO实验的后续项目,旨在利用液体闪烁体探测器对中微子进行更精确的测量,
以及测量二中微子双贝塔衰变(2$\nu\beta\beta$)的半衰期和寻找一个极其稀有的事件——
无中微子双贝塔衰变(0$\nu\beta\beta$)\cite{albanese2021sno+}。
SNO+实验采用了液体闪烁体作为探测介质,具有较高的光产额,
可以有效地提高中微子的探测效率。
液体闪烁体探测器相比传统的水切伦科夫探测器,具有更高的能量分辨率和更好的时间分辨率,
因此可以用于探测低能中微子事件。但是,由于液体闪烁体退激发产生的闪烁光产额远大于切伦科夫光,
导致在进行事件的重建时相比于传统的水切伦科夫探测器更加困难。
因此,合适的事件重建算法在大型液体闪烁体探测器中显得尤为重要。

传统的事件重建算法主要基于最大似然估计的方法,并通过对探测器响应的模拟和拟合来实现事件的重建。
这些算法通常需要对探测器的和物理过程进行详细的建模,面对噪声和背景事件很容易被影响。
近年来,深度学习技术的发展,由数据驱动的方法在事件重建中逐渐显现出其优势。
深度学习方法通过对大量实验数据的学习,自动提取特征并进行事件重建,具有更好的适应性和鲁棒性。

但是,深度学习方法在事件重建中的应用仍然有需要关注的地方。一般来讲,深度学习训练出的模型在进行预测的时候,
给出的结果是一个单值,而不是一个分布,这使得模型在进行预测的时候缺乏不确定性评估。
同时,模型本身进行预测时也会带来一定的不确定性。模型的不确定性以及数据本身的统计不确定性都会
影响到模型的预测结果。
因此,如何对深度学习模型进行不确定性评估,并将其应用于事件重建中,是一个重要的研究方向。


\section{研究目的}

本论文研究目的为研究基于深度学习的事件重建方法,并对其进行不确定性评估。
我们将采用深度学习模型对液体闪烁体探测器中的中微子事件进行重建,并对模型的预测结果进行不确定性评估。
通过使用不确定性评估方法,分析其在事件重建中的应用效果,为后续的实验数据分析提供参考。

\section{论文内容安排}

本文的第一章首先介绍了中微子物理学的背景和研究现状,第二章重点介绍了中微子的发现历程和双贝塔衰变现象;
第三章介绍了深度学习以及其带来的不确定性分析;第四章介绍了SNO+实验,重点介绍在该实验中传统的重建方法;
第五章介绍了基于深度学习的事件重建方法,并对其进行不确定性评估;最后一章总结了本文的研究工作,
并说明了未来可以在SNO+实验进行进一步研究并实施的方向,方法。